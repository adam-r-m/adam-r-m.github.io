
\documentclass[12pt]{article}

\usepackage[paperwidth=8.5in,paperheight=11in,top=.75in,left=.75in,right=.75in,bottom=.75in]{geometry} 


\usepackage{amsmath,mathtools,latexsym, ifthen, calc}
\vfuzz10pt % Don't report over-full v-boxes if over-edge is small
\hfuzz10pt % Don't report over-full h-boxes if over-edge is small
\usepackage{xspace}
\usepackage{abbrevs, nth}
\usepackage[nodisplayskipstretch]{setspace}
\usepackage[compress]{natbib}
\usepackage{txfonts}

\bibpunct{(}{)}{,}{a}{}{,}
\setcitestyle{numbers,square}

\defcitealias{crossleyhumberstone77}{Crossley-H'stone 1977}
\defcitealias{davieshumberstone81}{Davies-H'stone 1981}


%Murray-addd (editing)
\usepackage{color, soul}
\usepackage{comment}


\begin{document}


‘Modal’ phenomena pertain to what is necessary or possible.  On a canonical
analysis, modality is explicated in terms of ‘possible worlds’: necessities
hold (are true) in all of them, possibilities in some; while ‘mere’
possibility is distinguished from factuality by identifying a unique ‘actual
world’, such that what is true in it is exactly what is true ‘full stop’ (we
‘occupy’ it), with any other possible world verifying at least some falsehood.
On this analysis, reasoning about modality involves ‘consideration’ of
possible worlds; but a contrast in \ul{\textbf{modal perspective}} in such
reasoning has been widely recognized since the late 1970s, between considering
a world ‘as counterfactual’ (thinking about it ‘from the outside’) and ‘as
actual’ (‘from the inside’: pretending we occupy it). 

Metaphysics has long invoked the possible-worlds analysis; but (in striking
contrast with epistemology) has made scant use of this contrast in modal
perspective, with a pervasive tropism toward the \ul{consideration of
possibilities as-counterfactual}.  \ul{By contrast, our} proposed research
program \ul{will} demonstrate that appreciation of \ul{the as-actual dimension
of modal perspective generates novel (and systematically unified)} solutions
to a diversity of long-standing metaphysical puzzles. \ul{Headway on this
research program has already been made in our recent collaborative
investigations. Our objective during the granting period will be to round out
and extend this initial research to a full treatment of the metaphysical
significance of modal perspective. We envisage the project as generating at
least one-half dozen singly- or jointly-authored research articles,
collectively establishing the foundations for an extended monograph-length
development of the perspectivist program.} 

The puzzles \ul{to be} addressed exhibit a common tension: for \ul{many claims} M of
metaphysics, \ul{certain data supports a \textbf{stability} thesis} (M holds, or
else fails to hold, in absolutely all possible worlds) \ul{while certain other
data supports a verdict of \textbf{dependence}} (M somehow varies in truth from
world to world \ul{as a function of underlying contingent circumstances}).
With only the as-counterfactual perspective \ul{at our disposal}, the tension
is genuine: if M holds (or else fails) in absolutely all worlds
as-counterfactual, M has (of course) no scope for variation across worlds
as-counterfactual. But invoking the as-actual perspective dissipates any
tension: fixing a world \ul{as-actual}, M holds (or else fails) in absolutely
all worlds as-counterfactual (stability); but varying the as-actual world can
vary which of `hold' or `fail' it is (dependence).

To illustrate: `Chisholm's Paradox' of material artifact origins poses a
long-familiar putative tension in intuitive judgement.  \ul{Suppose a} hammer,
H, is built from an iron head and an oak handle. Intuitively: (\ul{M})
building H from a lead head and an elm handle is absolutely (`metaphysically')
impossible; \ul{plausibly}, because M concerns what is \ul{absolutely}
impossible, \ul{its truth} is invariant from world to world (stability).
\ul{And yet, equally intuitively}, H built from iron+elm \ul{(or lead+oak}) is
possible; and `in' an iron+elm-world, by parity of reason, lead+elm is
possible \ul{(not-M)}: so M does, after all, vary from world to world
(dependence). Solution: \ul{the putative tension is generated only by a tacit
`shift' of perspective mid-stream in our modal reasoning: the intuitive
stability of M occurs within the as-actual perspective upon H built from
iron+oak; whereas the `parity of reason' supporting dependence occurs
within the scope of an as-actual perspective upon H's origins in iron+elm}.  

Aspects of the \ul{research program cut across} several subregions of
philosophy. (1) A canvas of the scope of further metaphysical puzzles falling
to this strategy, and their detailed analysis: preliminary work suggests a
remarkable breadth, covering, among others, puzzles in ontology \ul{(and its
logic)} as well as in metaphysics of science. (2) Matters in philosophy of
language including both ‘descriptive semantics’ (the formal analysis of
particular linguistic constructions) and ‘metasemantics’ (the articulation and
interpretation of a framework within which descriptive semantics is best
pursued). (3) A methodology of `bounded naturalism' in metaphysics, 
in which the limits
of possibility are fixed in tandem by contingent, categorical circumstances
together with absolute and invariant principles of metaphysical logic.}  
(4) Historical investigation, to account for the surprising neglect
of modal perspective in metaphysics, into both the development of specific
subliteratures and more general tendencies in the analytical tradition. 

\eject

%Move to path

\bibliographystyle{newbenj}
%\bibliography{hmw}








\end{document}
