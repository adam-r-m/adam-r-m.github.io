\documentclass[12pt]{article}

\usepackage[paperwidth=8.5in,paperheight=11in,top=.75in,left=.75in,right=.75in,bottom=.75in]{geometry} 


\usepackage{amsmath,mathtools,latexsym, ifthen, calc}
\vfuzz10pt % Don't report over-full v-boxes if over-edge is small
\hfuzz10pt % Don't report over-full h-boxes if over-edge is small
\usepackage{xspace}
\usepackage{abbrevs, nth}
\usepackage[nodisplayskipstretch]{setspace}
\usepackage[compress]{natbib}
\usepackage{txfonts}

\bibpunct{(}{)}{,}{a}{}{,}
\setcitestyle{numbers,square}

\defcitealias{crossleyhumberstone77}{Crossley-H'stone 1977}
\defcitealias{davieshumberstone81}{Davies-H'stone 1981}


%Murray-addd (editing)
\usepackage{color, soul}
\usepackage{comment}


\begin{document}


% \subsection*{Detailed description}




\subsubsection*{Detailed Description}


\paragraph{Context} The context for our project consists, centrally, of (I) the
well-known relational (`possible-worlds') semantics for modal propositional
logic and its various extensions to modal predicate logic, a technical
resource first flourishing from the late 1950s and over the course of the
1960s. The context is filled out by two branches from this central component,
each of them first flourishing from the late 1960s over the course of the
1970s: (II) the `modal metaphysics' tradition of reading metaphysical
interpretations into the technical resources from (I) and/or invoking these
resources to shed light on pre-existing metaphysical questions; and (III) the
`modal pragmatics' tradition of extending the resources from (I) to
accommodate the broadly pragmatical phenomena of context-sensitivity and
speech-act content.

\smallskip{}

(I) Contemporary research into the logic of modalities commenced in the
opening third of the twentieth century, with CI~Lewis's syntactic exploration
in \citep{lewislangford32} of a variety of systems of intensional logic;
roughly concurrently, the conception of a possible-worlds semantics was
nascent in Wittgenstein's \emph{Tractatus} \citep{witttlp}, with its informal
theory of sentence-meanings as truth-conditions on possible states of affairs.
The 1940s witnessed Carnap's gradual accretion
\citep{carnapis,carnap46,carnapmn} of certain central elements of a
\emph{Tractatus}-inspired possible-worlds semantics for an elementary modal
predicate logic (with its propositional fragment the strongest non-trivial
modal propositional system, Lewis's $\mathsf{S5}$, and the modality
interpreted as a universal quantifier over worlds); alongside this early
semantical work came Ruth Barcan's \citep{barcan46b,barcan46a,barcan47}
contributions to the development of axiomatic modal predicate logic.
Over the course of the 1950s, efforts by a number of authors would
gradually integrate these semantical and syntactical strands, culminating
in the near-simultaneous publication by Hintikka \citep{hintikka},
Bayart \citep{Bayart1959-BAYQDL}, and Kripke \citep{kripke59} of
completeness proofs for quantified $\mathsf{S5}$, and later by Kripke
\citep{Kripke1963-KRISAO} for many sub-$\mathsf{S5}$ propositional
systems.\footnote{For detailed analysis of this period, see
\citep{copeland02}: signal works discussed are
\citep{McKinsey1948-MCKSTA,jonsson-tarski,jonsson-tarski-2,vonwright53,meredithprior56,montague60,kripke59,kripke63,Hintikka1961-HINMAQ,Lemmon1966-LEMASF-3}.}

The apparatus of relational semantics for propositional modal languages is
robust, and familiar in philosophy: a `frame' pairs a set of `possible worlds'
with an `accessibility relation' over them; a set of frames $\mathbf{F}$
characterizes a propositional modal system $\mathsf{S}$ as the set of all
sentences true under any valuation in any world from any frame in
$\mathbf{F}$, where, in particular, for a frame with accessibility 
relation $R$, $\Box\phi$
is true in world $w$ just if $\phi$ is true in world $w'$ whenever $R(w, w')$.
The strength of $\mathsf{S}$ varies (in a richly-explored manner) with the
constraints on `accessibility': so the set of all equivalence frames (with $R$ an
equivalence relation) characterizes the system $\mathsf{S5}$ (the normal modal
system with axioms \emph{T}: $\Box p \supset p$; \emph{4}: $\Box p \supset
\Box\Box p$; and \emph{B}:  $p \supset \Box\Diamond p$); while the set of all
reflexive, transitive frames characterizes the system $\mathsf{S4}$ (with
axioms \emph{T} and \emph{4}); and so forth.\footnote{Overviews of these
issues abound; a small selection:
\citep{chellas80,Cresswell1996-CREANI-3,blackburn-ml}.} 

Semantical analysis of modal predicate languages is perforce more complex, its
robustness attenuated proportionately, with the most straightforward
`Carnapean' implementation impeded by immediate philosophical worries. That
implementation enriches relational semantics with an individual domain, while
analyzing $Fa$ as true in $w$ just if $F$ is satisfied in $w$ by the
denotation of $a$ (perhaps, following Carnap, by the $w$-value of the
`individual concept' assigned to $a$), and $\forall x\,Fx$ as true in $w$ just
if $F$ is satisfied in $w$ relative to the entire domain: but these analyses
validate $\Box\exists x\,x = a$ (`necessarily, $a$ is something') and
represent quantifiers and modals as freely commutable, yielding the
so-called `Barcan Equivalence' of $\Box\forall x\,\phi$ and $\forall
x\,\Box\phi$---thus predicting the putatively unattractive necessity of
existence and nonexistence. Weaker systems require more intricate
characterizing semantical analyses; an assortment have been proposed, each
facing generally recognized shortfalls of either technical or conceptual
adequacy. Such shortfalls are mirrored in attendant complications to the
underlying (axiomatically characterized) logic: `classical' modal predicate
logic in the style of Barcan and Carnap is sound and complete relative to the
straighforward semantical implementation noted above, and thus includes both
$\Box\exists x = a$ and the Barcan Equivalence as theorems.\footnote{For a
systematic, opinionated overview of the technical options, see
\citep{garson84,Garson2005-GARUQM-2,Garson2006-GARMLF}, and \citep[Ch.\
3]{Williamson2013-WILMLA-2}.} 

\smallskip{}

(II) Shortly following (or sometimes concurrently and in concert with) these
technical developments, theorists came to notice a variety of applications,
particularly in the analysis of the conceptual repertoire of the natural
sciences (e.g., laws of nature); and also a variety of interpretive challenges, pertaining especially
to ontology (among these latter are `Chisholm's Paradox' from the Summary
description, and also the `Barcanite' issue just now briefly sketched). This
cluster of applications and challenges has over the interim coalesced into an
identifiably, if somewhat diffusely, interconnected literature on `modal
metaphysics', with its highly influential touchstone works Kripke's
\citep{kripke72} and DK~Lewis's \citep{lewis86} (notably also:
\citep{Plantinga1974-PLATNO,VanInwagen1990-VANMB,sider01,finetime,Hawthorne2006-HAWME}).
We now sketch three such challenges fitting the
\textsc{stability}/\textsc{dependence} structure noted in the
Summary---severally cutting across the syntactic and semantic strands
addressed in (I), and collectively illustrating the broad significance of
\emph{modal perspective}.

(A) An \emph{undermining} puzzle about laws of nature. A \textsc{stability}
principle here would exclude inter-world variability in the truth-value of
$\varphi$ whenever it is a \emph{law of nature} that $\varphi$. 
On its behalf, \emph{necessitarians}
cite the explanatory power of appeals to law: if we answer the question `why,
if it goes up, will it come down?'\ with `it is a law of nature that what goes
up comes down', necessitarianism virtuously avoids the peculiar further
question `OK, but what distinguishes this case from a case in which 
it is a law of nature that what goes up comes down, but things can go up
without coming down?'.\footnote{For defense, see \citep{loewer96,loewer12};
and in the context of the undermining puzzle, \citep[p.~247]{finevn}.} If, as
is generally agreed, the logic of metaphysical necessity is $\mathsf{S5}$,
this necessitarian reasoning excludes such variability.

But a powerful case for \textsc{dependence} (admitting such variability) comes
from the idea that the laws of nature should be sensitive to the `categorical'
facts on the ground---on pain of being `brute', inexplicable sources of
constraint without reciprocal responsibility to the categorical---an idea
incorporated by leading theories of laws: perhaps the laws are the simplest,
strongest systematization of the categorical facts (`Humean' laws); or perhaps
properties carry their causal powers essentially, and the laws are generated by the
categorical fact that exactly \emph{these} properties are instantiated
(`Aristotelian' laws). But this sensitivity predicts
\emph{undermining}---actual categorical facts yielding actual laws permitting
categorical facts yielding different laws permitting different categorical
facts: perhaps \emph{richly}, permitting actually forbidden possibilities; or
perhaps \emph{leanly}, forbidding actualities. Within the literature,
\textsc{dependence} theorists forego the explanatory power of
necessitarianism; though the prospect exists of `accessibilism' (as under (B))
for metaphysical necessity (abandoning \emph{transitivity} for \emph{rich}
undermining, or \emph{symmetry} for \emph{lean} undermining).\footnote{Humean:
\citep[sec.~3.3]{lewis73}; Aristotelian: \citep{shoemaker80}. Humean
undermining: \citep[p.~20]{lewis86}; Aristotelian undermining:
\citep[sec.~3.1]{carroll94}, \citep[p.~244--5]{finevn}; `undermining':
\citep[p.~xv]{lewis86intro} on \citep{lewis80}---compare
\citep[p.~246n16]{finevn}.} 

(B) \emph{Chisholm's Paradox} of individual essence. Recall the `Summary'
example of hammer $H$ actually built from iron+oak: the \textsc{stability}
principle moves from \emph{$H$ could (metaphysically) not have been built from
lead+elm} and its metaphysical necessity, along with the $\mathsf{S5}$ logic
of the latter to the unavailability of any possible world verifying \emph{$H$
could have been built from lead+elm}; while the \textsc{dependence} principle
proceeds from \emph{$H$ could have been built from iron+elm} together with
reasoning by parity between actuality and an iron+elm-world (single-part
replacements are possible in both or neither) to the truth at an
iron+elm-world of \emph{$H$ could have been built from lead+elm}.

Within the literature, some resolutions of the conflict abandon
\textsc{stability}: whereas with (A), the \emph{it is a law of nature}
operator may be weakened below $\mathsf{S5}$, the only operator available for
such weakening in (B) is \emph{metaphysical necessity}. One such strategy
adopts an `accessibilist' structure for metaphysical necessity, declaring a
world in which $H$ is made from lead+elm to be actually `inaccessible' (and
thus impossible) but `accessible' from an iron+elm-world (and thus `possibly
possible'). But a sub-$\mathsf{S5}$ logic is also attainable via a
`counterpart' semantics, according to which the shifts in index-world induced
by a modal operator can adjust the individual concept assigned to an embedded
term: then we may allow that, contrasting the actual world $a$ with an
iron+elm-world $w$, indexing to the one rather than the other shifts the
individual concept for `$H$' so that, in a lead+elm-world $w^{\prime}$,
indexing to $w$ yields (in $w^{\prime}$) a lead+elm aggregate but indexing to
$a$ does not---thereby verifying in $a$ both \emph{$H$ could not have been
built from lead+elm} and \emph{possibly, $H$ could have been built from
lead+elm} (the former because the index for `could' is fixed to $a$, and the
latter because the embedding modal shifts it to $w$).\footnote{Accessibilism:
\citep{chandler76,salmon81,salmon89}; objections: \citep[pp.~246--8]{lewis86},
\citep[sec.~8.3]{williamson90}. Counterpart theory:
\citep{lewis68,lewis71cptb,Lewis93cpxap}; as a semantics:
\citep{fara08,fara12}, \citep[sec.~3.2.2]{hmw}; objections:
\citep{Hazen1979-HAZCSF,Woollaston1994-WOOCTA-3,Fara2005-FARCAA,Fara2009-FARDH};
and CP: \citep{forbes84}, \citep[p.~248]{lewis86}; objections:
\citep[sec.~8.3]{williamson90}.}

Other resolutions abandon \textsc{dependence}: `mereological essentialism'
rejects \emph{$H$ could have been built from iron+elm} (dually,
`antiessentialism' admits \emph{$H$ could have been built from lead+elm});
while a `sharp cutoff' view denies any legitimacy to reasoning by parity from
an iron+elm-world: folks there accept \emph{$H$ could have been built from
lead+elm}, but are brutely wrong in doing so.  Still other resolutions
preserve both \textsc{stability} and \textsc{dependence} by positing a tacit
`shift' in \emph{de re} subject-matter as attention tracks $H$ from $a$ over
to the iron+elm world $w$: then we may allow that our focus in $w$ is not $H$
after all, but rather a materially indiscernible iron+elm aggregate
$H^{\prime}$ that $H$ entirely overlaps, and which could be made from
lead+elm---thereby explaining the illusion (in $a$) that it is possible for
$H$ to be possibly made from lead+elm.\footnote{Essentialism:
\citep{chisholm67,Chisholm1973-CHIPAE,Chisholm1975-CHIMES,Zimmerman1992-ZIMAAF,Steen2008-STECCC-2};
objections:
\citep{Plantinga1975-PLAOME,Wiggins1979-WIGME,Kleinschmidt2014-KLEMAL};
antiessentialism: \citep{Mackie2006-MACHTM}. Sharp cutoff:
\citep{williamson90}. Shiftiness:
\citep{Dorr2021-DORTBO-2,Leslie2011-LESEPA,RobertsonIshii2022-ROBEBT,yagisawa}.} 

(C) A \emph{Barcanite} challenge about the necessity of existence and
nonexistence. Very briefly: can there be inter-world variability in
truth-value of statements of ontology (to the effect, perhaps, that the
cardinality of the existents is exactly such-and-such; or that some individual
(designated `de re') exists; or that some essential, uniquely individuating 
predicate is satisfied or not)? On behalf of \textsc{stability} is the broad
abductive support attaching to the simple Barcan--Carnap logic and semantics
on which there could \emph{not} be such inter-world variability in the
existents. But on behalf of \textsc{dependence} is the thought that, for a
certain human $d$, whether $d$ exists depends on whether their parents engaged
in appropriate begetting activity: in possibilities where their parents do not so
engage, matter is not ordered in a way sufficient for $d$ to exist as a human;
and if $d$ is a human, $d$ is essentially so, and could not possibly exist as
`contingently nonconcrete'.\footnote{`Barcanists' siding with
\textsc{stability}:
\citep{zalta88,linskyzalta94,linskyzalta96,williamson98bp,williamson13}.
Barcanism and the `contingently nonconcrete': \citep{linskyzalta96},
\citep[p.~266]{williamson98bp}, \citep{williamson13}. Contingent
\emph{existence}: \citep[p.~257]{garson84}, \citep[p.~258]{williamson98bp},
and cites in \citep[ch.~1]{williamson13}; \emph{nonexistence}:
\citep[pp.~65--6]{kripke63}, \citep[p.~258]{williamson98bp}. The canonical
`domain relativization' semantics for \textsc{dependence}:
\citep{kripke63,stalnaker94}; complaints:
\citep{garson84,Garson2005-GARUQM-2,williamson98bp,williamson13}; taxonomy of
other strategies: \citep[p.~250]{garson84}; `individual concepts':
\citep{thomason69,Garson2005-GARUQM-2}. An approach with some affinity to RMM:
\citep{fineneno}.}

\smallskip{}

(III) The end of the 1960s witnessed breakthroughs in the application of
intensional semantics to broadly `pragmatic' phenomena, particularly through
the isolation of an important distinction in the broader category of
`context-sensitivity', a distinction giving rise to certain logical
peculiarities. The fundamental technical resource for accommodating these
phenomena, developed in the early 1970s, was a `two-dimensional' (2D)
foliation of the earlier intensional semantics. Later in the decade,
epistemologically-minded theorists located within the expanded approach the
representational import of its `horizontal' and `diagonal' projections; this,
in turn, led to the discovery of the `as-counterfactual'/`as-actual' contrast
in modal perspective mentioned in the Summary.

To expand. A long-standing interest of logicians of tense was the
`indexicality' of temporary sentences like \emph{the sun is in eclipse} in
interaction with temporal operators like \emph{always} or \emph{henceforth}:
analogously to the world-semantics for a contingent sentence like
\emph{Hamilton was born on Nevis} and the quantificational action of modals, a
moment-semantics relativizes truth-value to moments of time (perhaps under a
certain ordering), and analyzes \emph{always} as a universal quantifier over
moments. The category of `indexicality' was unsettled by Kamp's 1967 discovery
(\emph{avant le lettre}) of the `rigidifying' action of \emph{now}: the
apparent logical validity of \emph{$\phi$ just if now, $\phi$} is not
sustained through embedment under \emph{always}; a generalization to the modal
case, swiftly observed by DK~Lewis, involves the rigidifying \emph{actually}:
the apparent logical validity of \emph{$\phi$ just if actually, $\phi$} is not
sustained through embedment under \emph{necessarily}.\footnote{`Indexicality':
\citep{montagueprag,montaguepragil,lewis70gs}. `Rigid': \citep{kripke72};
rigidification: \citep{kamp68,kamp71,Prior1968-PRIN-3,lewis70}.}

The early 1970s subsumed proposals of Kamp and Lewis under a general
`double-indexing' framework assigning truth-values along two `dimensions' and
providing a `diagonal' or `real-world' analysis of logical consequence. For
modals, truth is relativized to both an `index'- and a `context'-world; the
data of the embedment puzzle are then explained thus: relative to worlds $i$
and $c$, \emph{necessarily, $\phi$} is true just if for every world $i'$,
$\phi$ is true relative to $i'$ and $c$, while \emph{actually, $\phi$} is true
just if $\phi$ is true relative to $c$ and $c$; and $\phi$ is logically valid
just if (on all frames and valuations) for every world $c$, $\phi$ is true
relative to $c$ and $c$.\footnote{Double indexation: \citep{segerberg73};
diagonal consequence:
\citep{Vlach1973-VLANAT,kaplan77,fritz11,Fritz2014-FRIWIT-2}; opposition:
\citep{crossleyhumberstone77,davieshumberstone81,Hanson2006-HANANA-5,Hanson2014-HANLTI}.
`Index'/`context': \citep{lewis80icc}.}

Later in the decade, various theorists noted an affinity to, and thus a
prospect of interpreting, Kripke's dissociation of metaphysical necessity from
apriority: \emph{$\phi$ just if actually, $\phi$} is predicted to be logically
valid and thus presumably apriori---but with its embedment under
\emph{necessarily} not valid, is predicted to be not metaphysically necessary;
while the above apriority suggests that \emph{actually, $\phi$} has the same
epistemic status as $\phi$, so that both are aposteriori if either is---but
\emph{actually, $\phi$} is predicted to be noncontingent.\footnote{2D and
Kripke: \citep[pp.~172--3]{lewis75ll},
\citep{kaplan77,stalnaker78,Evans1979-EVARAC,davieshumberstone81}.}

Late in this literature, `consideration as actual' was coined to label
reasoning about truth-values along the diagonal; during a mid-1990s resurgence
of interest, `consideration as counterfactual' was applied to reasoning about
truth-values along the horizontal. This terminology drove an eventual
explanatory theory of the Kripkean dissociation: what we care about in valid
reasoning is security against error about how things \emph{are} (actually);
while metaphysical necessity is manifestly concerned with how things
\emph{could be} (counterfactually)---thus, the dissociation stems ultimately
from a distinction in \emph{modal perspective}.\footnote{`As actual':
\citep{davieshumberstone81}; `as counterfactual':
\citep{chalmers96,jackson98}; resurgence:
\citep{jackson94,lewis94,blockstalnaker99,stalnaker01,chalmersosi,chalmersf2ds,gendlerhawthorne02}.} 

\bigskip{}

\noindent \textbf{Objectives and methodology} \, \,  Despite the roughly
concurrent appearance and early development of components (II) and (III) of
the context, each long unfolded in large part or entirely in isolation from
the other.  Their mutual relevance, as sketched in the Summary, is that each
puzzle in (II) can be resolved by appeal to the distinction in modal
perspective from (III): to wit, it is variation in the world considered
\emph{as counterfactual} which displays the putative \textsc{stability},
whereas the putative \textsc{dependence} appears varying instead the world
considered \emph{as actual}---so the appearance of conflict is engendered by
an equivocation of modal perspective. This observation was to our knowledge
first published in 2012 by Murray and Wilson \citep{murraywilson12}; what
has since come to be known as the `Relativized Metaphysical Modality' (RMM)
program develops the observation in service of a systematic reassessment of a
variety of subregions of metaphysics. Hellie joined the RMM program as
the co-supervisor (with Wilson) of Murray's dissertation \citep{murray17}, and
was the primary drafter of the overview \citep{hmw} (the puzzles in (II) are
expanded on there in sec.~3, and technicalities of RMM resolutions sketched in
sec.~4). Responses to RMM have since been published by Yagisawa
\citep{yagisawa}, Roberts \citep{Roberts2020-ROBTMA-11}, 
and Dorr, Hawthorne, and Yli-Vakkuri
\citep{Dorr2021-DORTBO-2}, among others.\footnote{Other reactions to RMM include 
\citep{LamForthcoming-LAMAIP-2,Clarke-Doane2019-CLAMAA-14,Bassford2020-BASART-3,Glazier2017-GLATDB-3,
Furtado2020-FURSAT}.} 

The objective of our proposed research is to contribute to the rounding out
and completion of the RMM program. The considerations involved are broadly
distributable into: (IV) issues in metaphysics, including the metaphysics of
science; (V) questions in philosophy of language and philosophical logic;
and (VI) matters of historical development. Hellie's
relevant areas of  expertise are more weighted toward the technical and
historical, with contributions to the program falling primarily under (V)
and (VI), and in localized subregions of (IV). Murray's areas of expertise
tend towards the metaphysical and the logical, with contributions to the
program falling primarily under (IV) and (V), and associated subregions of
(VI).  Wilson's relevant areas of expertise lie broadly in metaphysics
(including metaphysics of science) and in metametaphysics (including the
epistemology of metaphysics), with contributions to the program thus falling
primarily under (IV) and relevant subregions of (VI).  During the granting
period, we propose to publish, individually and jointly, a series of papers
further developing the RMM program as it distributes across these respective
areas.  This research will then fold into a collaboratively authored book, the
aim of which will be to provide comprehensive coverage and defense of the RMM
framework. 

\smallskip{}

(IV) An initial aspect of the metaphysical component involves a canvas of the
literature for further puzzles fitting the
\textsc{stability}/\textsc{dependence} schematism and potentially resoluble
along analogous lines: several examples stand out (preliminarily to a more
thorough survey). Closely resembling (A) is Lewis's `Big Bad Bug' concerning
objective chance, with its well-known conflict between the conceptual role of
chance (a source of \textsc{stability}) and an attractive metaphysics (a
source of \textsc{dependence}---indeed, the `undermining' considerations in
(A) are directly analogous to phenomena Lewis so labels).  Resembling aspects
of both (A) and (B) is a tension between views of metaphysical claims
(concerning, e.g., the nature of properties) as sensitive to each world's
deep categorical structure (thus motivating \textsc{dependence} as regards the
hypotheses of fundamental metaphysics), and conceptions of metaphysical
inquiry as aimed at the uncovering of universal (and so invariant) truths
concerning the nature of things (thus motivating \textsc{stability}).  A
direct application of (C) is a well-known tension (arguably tracing to the
\emph{Tractatus} case for the necessity of objects, and thus their simplicity,
and thus a logically perfect language) between the conception of propositions
as sets of worlds (motivating \textsc{stability} concerning the existence of
propositions) and the thought that some propositions are `object-involving' or
\emph{de re} (motivating \textsc{dependence}).\footnote{(A):
\citep{lewis80sg,lewis86intro,lewis94debugged} and, inter alia,
\citep{Hall1994-HALCTG,Thau1994-THAUAA,Arntzenius2003-ARNOWW,Loewer2004-LOEDLH,Briggs2009-BRITAO,Pettigrew2013-PETWCN,Caie2015-CAICIT-3}.
(A + B): \citep{Cameron2007-CAMTCO-2,parsons2013-PARCCA-9, Rosen2006-ROSTLO-2,
Miller2009-MILDCI-3} (contingentism);
\citep{Sider1993-SIDVIA,fineem,Fine2011-FINWIM,Schaffer2007-SCHFNT}
(necessitism).  (C) \citep[2.02--2.0212]{witttlp}; inter alia,
\citep{Prior1967-PRIPPA-2,stalnaker12}, and citations in
\citep{Fritz2016-FRIPC-2}: Murray describes the RMM treatment in
\citep{murray-pdps}.}

A further aspect involves the development of an underlying, explanatory
picture: how can it be that what is metaphysically necessary makes, in some
way, an ineliminable appeal to how things actually are? In the present
context, the 2D apparatus is naturally read as parceling out the component
inputs to our reasoning about metaphysical necessity into \emph{specific
facts} and \emph{general principles}---the former component associated with
the categorical truths about the `contextual' world (the `as-actual'
coordinate), and the latter with a system of categorical-to-modal conditionals
(those evaluated as true in every `diagonal' coordinate pair). The overall
image is of a sort of `bounded naturalism' in metaphysics, in which the limits
of possibility for the world (so to speak) project from the (arbitrary, brute)
condition of nature, along (absolutely fixed, principled) vectors of logic.
This image finds a middle ground between a pure naturalism, on which questions
of metaphysics (when sensible) are handed over to natural science; and the
more `apollonian' approach of traditional modal metaphysics, in which
actuality is constrained by, but has no reciprocal input into, the limits of
possibility.\footnote{This polarity is typically manifest as an `ethos'
underlying detailed work, with certain works more representative of an `ideal
type': \citep{Fine2011-FINWIM,Hawthorne2011-HAWHMA,williamson13} (apollonean);
\citep{Ladyman2007-LADETM,Price2007-PRIQN} (naturalist). Bounded naturalist
sympathies appear in, e.g., the case from quantum entanglement to priority
monism in \citep{Schaffer2010-SCHMTP-2}, and in the case from Minkowskian
spacetime to perdurantism in \citep{sider01}.}


\smallskip{}

(V) Making sense of this bounded naturalism requires delicate follow-on
investigations in philosophy of language, in order to thread between a dilemma
of complaints that (D) the true notion of `metaphysical' necessity would
better apply to our truth in all worlds as-counterfactual, at all worlds
as-actual; and that (E) the categorical-to-modal conditionals required to
state `general principles' are inexpressible.

(D) The complaint is developed efficiently using the \emph{undermining} puzzle
from (II-A). On the view in \citep[sec.~4]{hmw}, the actual world, $a$,
induces laws of nature verified at just the worlds in a set $L(a)$; among
these is an `undermining' world $u$ inducing different laws of nature,
verified at just the worlds in a distinct set $L(u)$. Suppose that this $L(u)$
contains a world $w$ not in $L(a)$, and thus verifying some sentence $\phi$
false at every world in $L(a)$: according to the nomological necessitarian,
this $\phi$ is metaphysically impossible. But why say this? The world $w$ is
right there, verifying $\phi$, after all.  Although the consideration of the
actual world $a$ as actual renders $w$ unavailable for consideration as
counterfactual, $w$ becomes available for consideration as counterfactual upon
consideration of the undermining world $u$ as actual.  On the traditional
understanding of metaphysical possibility as the `broadest' variety of
possibility, attained by lifting absolutely all restrictions on quantification
over worlds, it would seem that $w$ is metaphysically possible: after all, we
are speaking about it, apparently requiring our capacity to quantify over
it---jumps among directions of modal perspective notwithstanding.

Our response invokes Kaplan's famous prohibition on `monsters begat by
elegance': intensional sentential operators can shift only \emph{indexical}
coordinates (such as an as-counterfactual world); shifts in \emph{contextual}
coordinates (such as an as-actual world), are beyond the power of natural
language.\footnote{\citep[sec.~VIII]{kaplan77};  also
\citep{lewis80icc,Israel1996-ISRWMD,Schlenker2002-SCHAPF-4,santorio12,Rabern2013-RABTMQ,stalnaker14}.}
But the complaint in (D) appeals to `monstrous' metaphysical modal operators:
the troublesome world $w$ is beyond the set $L(a)$ and so no modal operator
can quantify over $w$ with the contextual world fixed to $a$; upon
consideration of $a$ as-actual it would require a monstrous shift of the
contextual world to $u$ in order to bring $w$ within the domain of modal
quantification.

Why bar monsters? Briefly, an empirical motivation sees the Kamp/Lewis discrimination
of `indexical' from `contextual' dependence as reflecting a real contrast in
semantical argument structure: the former a kind of unsaturatedness amenable
to control by compositional, language-internal mechanisms, with the latter
reserving control to language-external, contextual factors. 
For a logical
motivation, allowing \emph{all} monsters admits a `diagonal necessity'
operator $\boxbslash$ with $\boxbslash\phi$ a tautology if
$\phi$ is true down the diagonal, and otherwise a contradiction. 
However, and drawing upon an observation of Williamson's, the interaction of 
$\boxbslash$ with phenomena of rigidification 
would collapse natural-language truth conditions into trivialities, short of
\emph{ad hoc} restrictions of motivation or expressive power. Securing against
such collapse thus requires a general ban on `monstrous' operators like
$\boxbslash$.\footnote{Empirical:
\citep[pp.~31--2]{lewis80icc}. $\Act$: \citep{williamson09ca}. Collapse:
compare the \citep{church43c} `slingshot' against \citep{carnapis}; and more
generally \citep{Neale2001-NEAFF}.} 


(E) Over to the other horn: barring monsters would seem to bar the
categorical-to-modal conditionals articulating `general principles', rendering
bounded naturalism inexpressible. After all, we grasp these principles by
shifting the world considered as actual; so a conditional apt to state such
reasoning would be along such lines as `if things are actually such that
$\phi$, it is metaphysically necessary that $\psi$'. But in order to have the
intended result, the \emph{if}-clause must somehow shift the world of the
\emph{context} against which the consequent is evaluated; and, having barred
monsters, it cannot do so through the ordinary semantic activity of an
intensional operator---so unless (indicative) \emph{if}-clauses are somehow
semantically extraordinary, the intended result is unobtainable, and bounded
naturalism is inexpressible.  Our response is that \emph{if}-clauses
\emph{are} semantically extraordinary. An ordinary intensional operator---a
modal or temporal operator, or the \emph{if}-clause as analyzed by
Stalnaker/Lewis---adjusts a coordinate against which its operand is evaluated
during semantic composition. In contrast, the shift of context associated with
an \emph{if}-clause does not occur during semantic composition, but is instead
a `postsemantic' effect, occuring only once a fully composed semantic value
must project a certain propositional content against a specified context
$c$.\footnote{`Ordinary': \citep[pp.~27--31]{lewis80icc}. `Stalnaker/Lewis':
\citep{stalnakeratoc,lewis73}. `Postsemantic': \citep{macfarlane14}.
`Selection': \citep{stalnakeratoc,Cariani2018-CARWDB-6}.}
 
\smallskip{}

(VI) If the distinction in modal perspective has been widely discussed in
epistemology but scarcely noticed in metaphysics, why? (F) Technical barriers
include: (1) the logical tools exploited by opponents of \textsc{stability}
are all older than double indexation (though, contrastingly, a fully
\emph{semantic} counterpart theory is a relatively recent development); (2)
the various puzzles are themselves old; (3) pivotal figures Lewis and Kripke
are, respectively, by admission compartmentalized in thinking about double
indexation and largely silent about it; (4) the diagonal conception of
consequence coalesced later than double-indexation, and certain metatheoretic
oddities render it yet controversial; (5) monster-barring remains
controversial, with its motivation not generally appreciated; (6) standing
theories of the conditional are not adeqate to bounded naturalism. (G)
Conceptual barriers (more speculatively) include: conceptions of metaphysics
as `apollonian' or `absolute', requiring `objectivity' and excluding
perspectival input; or as hand in hand with logic, the latter envisaged as in
tension with perspective sensitivity.\footnote{This sketch is lightly expanded
upon in \citep[sec.~5]{hmw}.}


\eject

%Move to path

\bibliographystyle{newbenj}
\bibliography{hmw}
%\bibliographystyle{/home/adam/Dropbox/sshrc/HMW/newbenj}
%\bibliography{/home/adam/sshrc/HMW/hmw}








\end{document}
