\documentclass[12pt]{article}

\usepackage[paperwidth=8.5in,paperheight=11in,top=.75in,left=.75in,right=.75in,bottom=.75in]{geometry} 


\usepackage{amsmath,mathtools,latexsym, ifthen, calc}
\vfuzz10pt % Don't report over-full v-boxes if over-edge is small
\hfuzz10pt % Don't report over-full h-boxes if over-edge is small
\usepackage{xspace}
\usepackage{abbrevs, nth}
\usepackage[nodisplayskipstretch]{setspace}
\usepackage[compress]{natbib}
\usepackage{txfonts}

\bibpunct{(}{)}{,}{a}{}{,}
\setcitestyle{numbers,square}

\defcitealias{crossleyhumberstone77}{Crossley-H'stone 1977}
\defcitealias{davieshumberstone81}{Davies-H'stone 1981}


%Murray-added (editing)
\usepackage{color, soul}
\usepackage{comment}


\begin{document}



\paragraph{Knowledge mobilization plan}

The primary vector for knowledge generated by the proposed research is upwards
of eight individually- or jointly-authored articles and a collaboratively
authored monograph (working title: \emph{Modal Metaphysics Relativized}).  We
have already (in various groupings) produced three articles and a dissertation
on RMM, collectively garnering at least ten current or forthcoming citations.
Across each of the first three years of the granting period (Y1-Y3), we will
each produce at least one singly-authored or co-authored research article,
contributing to the further development of the RMM program as discussed in
(IV--VI) of the Detailed Description. The results of this research will then
be folded (in Y4) into preliminary chapter drafts for the monograph and a book
proposal, informally encouraged by Peter Momtchiloff of Oxford University
Press.  

\smallskip{}

Across Y1-Y3, \textbf{Hellie} will publish a number of single-authored journal
articles on relevant issues in philosophy of language and history of analytic
philosophy. In philosophy of language: 
(1a) `Rigidification and
monstrosity' develops the argument sketched in (V-D) of the Detailed
Description, for submission to \emph{Analysis}; (1b) `A postsemantic solution
to two triviality problems for conditionals' develops (V-E): this unavoidably
long paper is appropriate to \emph{The Philosophical Review} (and may require
splitting in two); (1c) `Why and whither postsemantics?' lays the foundation
for (V-E): each of \emph{Philosophical Studies} and \emph{Philosophy and
Phenomenological Research} has published work cited here. In history of
philosophy (VI-G), (1d) `The `apollonian' strand in analytic metaphysics'
discusses `perspective-freedom' in Lewis and Kripke, mixing history and
metaphysics characteristically for \emph{Mind}; and (1e) `Parametrization in
semantics' (to be co-authored with Murray) concerns (VI-F), and the historical
independence of metaphysics from the `modal pragmatics' tradition described in
(III).    

\smallskip{}

\textbf{Murray} will publish during this same period several articles in
metaphysics and philosophical logic: (2a) `Existence and
Nonexistence in Modal Perspectivism' develops the RMM solution to the
`Barcanite' puzzle of (II-C), and is suitable for publication in \emph{Erkenntnis}; (2b)
`Higher-order Necessitism' develops the quantified modal logic and metaphysics of properties and
propositions in connection with (IV-C), and is appropriate to the \emph{Journal of
Philosophical Logic}; and (2c) `Further Considerations on Chisholm's Paradox'
extends the RMM treatment of (II-B) to related puzzles of `modal tolerance',
and responds to recent critical discussion of RMM published in
\emph{Synthese}, appropriately for submission to the latter journal. 

\smallskip{}

Finally, \textbf{Wilson} will author a pair of articles in metaphysics: (3a)
`Metaphysical Skepticism, Relativized Metaphysical Modality, and A Posteriori
Metaphysical Necessities' develops (IV-A+B), and is suitable for
\emph{No\^{u}s}; and (3b) `Metaphysics as Bounded 
Naturalism' (to be co-authored with Murray) develops the underlying
explanatory framework of (IV), and is suitable for \emph{Mind}.  

\smallskip{}

We plan to complement the research conducted in each of Y1-Y3 with a
thematically-related conference. The first, in Y1 at Toronto, will deal with
perspectivism in language and logic; the second, in Y2 at Manitoba, will focus on
perspectivism in modal metaphysics; and the third, in Y3 at Toronto, will be an
invited workshop centered upon the broad themes of RMM and the proposed
monograph. 

\smallskip{}

A final dimension of research mobilization lies in graduate-level teaching. We
anticipate substantial graduate student involvement, both in conducting
literature surveys crucial to the development of the aforementioned papers,
and in each of the conferences and workshops described above. We have each
taught material relevant to RMM in our graduate seminars: Hellie in the
Toronto philosophy department in 2014, '18, '21, and '22; Wilson in the
Toronto philosophy department in 2012, '13, '14, and '22; and Murray in both
the Toronto and Manitoba philosophy departments in 2017, '18, and '21.  Our
project takes a fresh look at a wide variety of problems of metaphysics
through the underappreciated but technically well-understood phenomenon of
modal perspective: a number of philosophers have become tuned in to the
potential of the RMM program, and given its excellent PhD program the
department at Toronto is particularly well-situated to become a world center for its
investigation. That investigation naturally funnels into valuable
opportunities for research-level interaction at the graduate level. 



\end{document}


