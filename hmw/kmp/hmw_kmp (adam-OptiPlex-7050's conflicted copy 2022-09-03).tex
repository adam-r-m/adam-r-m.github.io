\documentclass[12pt]{article}

\usepackage[paperwidth=8.5in,paperheight=11in,top=.75in,left=.75in,right=.75in,bottom=.75in]{geometry} 

\usepackage{benj,abbrevs}

\usepackage{txfonts}

\begin{document}

\paragraph{Knowledge mobilization plan}


The primary vector for knowledge generated by the proposed research will be several articles and (my contribution to) a monograph. The monograph (working title \emph{Modal Metaphysics Relativized}), a comprehensive examination of the `Relativized Metaphysical Modality' (RMM) program, will be co-authored with Canadian philosophy professors Adam Russell Murray (Manitoba) and Jessica M.~Wilson (Toronto) (collectively: `HMW'). HMW (in various groupings) have produced three articles and a dissertation on RMM, already garnering at least five current or forthcoming citations. Informally encouraged by Peter Momtchiloff of Oxford UP to prepare a book proposal on RMM, HMW have worked up a tentative outline and distribution of tasks for the proposal and monograph.

Feeding into the RMM program and HMW's monograph, I will publish a number of single-authored journal articles on issues in philosophy of language and history of analytic philosophy. In philosophy of language (of interest independently of RMM, and discussing ideas antedating my involvement): (1) `Rigidification and monstrosity' develops the argument sketched in (V-D) of the Detailed Description, for submission to \emph{Analysis}; (2) `A postsemantic solution to two triviality problems for conditionals' develops (V-E): this unavoidably long paper is appropriate to \emph{The Philosophical Review} (and may require splitting in two); (3) `Why and whither postsemantics?' lays the foundation for (V-E): each of \emph{Philosophical Studies} and \emph{Philosophy and Phenomenological Research} has published work cited here. In history of philosophy (VI-G), (4) `The `apollonian' strand in analytic metaphysics' discusses `perspective-freedom' in Lewis and Kripke, mixing history and metaphysics characteristically for \emph{Mind}.

I post pre-prints of all papers on my University of Toronto website (\texttt{benj.ca}); I aim to publish in open-access venues whenever possible, and will follow the Tri-Agency Open Access Policy. I will present draft material at professional conferences, workshops, seminars, and invited talks: components of the project already workshopped include (3), used in most of my talks this decade; and (2), in a 2018 talk.

This May, I submitted to OUP-USA a contracted book MS, titled \emph{Out of This World: Logical Mentalism and the Philosophy of Mind} (I expect reports by December), of about a decade in the writing. Alongside this major primary endeavour, I have published over one article per year while a faculty member, in top venues: specialist and generalist journals (\emph{Philosophical Review}, \emph{Philosophical Studies}, \emph{Analysis}, \emph{Mind}, \emph{Philosophers' Imprint}, \emph{European Journal of Philosophy}, \emph{Inquiry});  invited annuals (\emph{Philosophical Perspectives}, \emph{Philosophical Issues}); high-profile collections (Oxford UP, Blackwell, MIT Press); and longer reference works in major handbooks (Routledge, Cambridge UP). I have given (or have been invited to give) talks at philosophy departments in 21 countries (among `top ten' departments: NYU, Oxford, Rutgers, Harvard, Pittsburgh, Michigan, Berkeley, Toronto); and many conferences, workshops and seminars. I expect that during the granting period, I will sustain this rate of output and quality of venue.

My work brings technical tools from logic and mathematics to general philosophical problems. An important part of my work, then, consists in distilling and presenting technical results for readers who are not specialists in the respective technical sub-fields. For this purpose and others, presenting and discussing my work with philosophers from many different sub-fields, and with researchers in cognate areas such as linguistics and mathematics, has been essential to developing my ideas and shaping how those ideas are framed. The travel funding included in the budget for this project will enable me to submit my work to a number of important professional conferences each year and so to collect this important feedback.

Finally, I will share the results of my research through teaching. I sat on the committee for Murray's 2017 PhD thesis on the project material from 2010, and as co-supervisor from 2015, and have taught associated material in graduate seminars in the Toronto philosophy department in 2014, '18, '21, and '22. The project takes a fresh look at a wide variety of problems of metaphysics through the mildly cognitively challenging but technically well-understood phenomenon of modal perspective: a number of philosophers have become tuned in to the potential of the RMM program, and I am excited to continue to sustain Toronto's graduate program as a world center for its investigation, by teaching our graduate students the technical underpinnings of the program and illustrating its capacities.



\end{document}


