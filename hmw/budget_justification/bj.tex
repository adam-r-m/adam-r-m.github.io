
\documentclass[12pt]{article}

\usepackage[paperwidth=8.5in,paperheight=11in,top=.75in,left=.75in,right=.75in,bottom=.75in]{geometry} 


\usepackage{amsmath,mathtools,latexsym, ifthen, calc}
\vfuzz10pt % Don't report over-full v-boxes if over-edge is small
\hfuzz10pt % Don't report over-full h-boxes if over-edge is small
\usepackage{xspace}
\usepackage{abbrevs, nth}
\usepackage[nodisplayskipstretch]{setspace}
\usepackage[compress]{natbib}
\usepackage{txfonts}

\bibpunct{(}{)}{,}{a}{}{,}
\setcitestyle{numbers,square}

\defcitealias{crossleyhumberstone77}{Crossley-H'stone 1977}
\defcitealias{davieshumberstone81}{Davies-H'stone 1981}


%Murray-added (editing)
\usepackage{color, soul}
\usepackage{comment}


\begin{document}

\subsection*{Budget justification}

\subsubsection*{Personnel costs}

We have requested funds to hire graduate students as research assistants in
each year of the project.  Hellie and Wilson have jointly requested a total
of  \$35,000 per year
to to hire two University of Toronto doctoral students---one working primarily
in metaphysics, and one working primarily in logic and language---as research assistants
over the full term of the project. The standard departmental rate for such
work at Toronto is \$17,500 per year.  Murray has requested \$16,000 per year
to hire two University of Manitoba MA students in each year of the project, at
the standard departmental rate of \$8,000. There are a number of
graduate students in our respective departments that have research interests
suitably related to the themes of the project. 

The central duty of our research assistants will be to help research
literature that is relevant to the project. As the relevant literature spans
various subfields of philosophy over much of the last half century, this work
will be of critical importance. Our research assistants will track down
references and write detailed reviews of the most important papers that they
identify. They will also compile master bibliographic files for all of the
sources relevant to the project, including many papers we have already
identified. This assistance will significantly increase the rate at which the
project proceeds. We will also rely on the research assistants for editorial
assistance. They will be responsible for manuscript preparation and
proofreading of articles when they are prepared for submission to journals,
and when the final proofs get returned for checking. Our research assistants
will also help with the preparation of the anticipated book manuscript
discussed in the project description.  In addition to contributing to the
research project, this work will help our research
assistants learn about the publication process, and will prepare them for
publishing their own work in the future.

Our regular discussions of the literature with our research assistants will
also very likely lead to opportunities for collaborative work. There are many
topics closely related to our project that we will encourage our research
assistants to explore. These discussions will potentially lead to
co-authored work with our students, or to independent research in the form of
a Master's or Doctoral thesis. 


\subsubsection*{Travel and subsistence costs}

We have budgeted a total of \$3825 per year for each team member to travel
within Canada to present results from the project, primarily at meetings of
the Canadian and Western Canadian Philosophical Associations.  We anticipate
that our individual expenses for such travel will be approximately \$1275 per
trip---an average of \$600 for airfare and ground transportation, \$450 for 3
nights in a hotel, and \$225 for three days of meals and incidentals
(reflecting a standard domestic per diem of \$75). Any leftover funds from
this category will be used for minor research related travel costs. In
particular they will be used for short trips to meet with scholars working on
related issues at other Canadian universities. We have also budgeted an
additional \$3825 per year covering one trip per team member between Winnipeg
and Toronto, to support periods of sustained collaborative work and
discussion, and for participation in the annual project workshops detailed in
the Knowledge Mobilization Plan. 

In years one and four of the project, we have budgeted a total of \$11,400 to
cover travel for each team member to two major professional conferences and
workshops in the US and Europe, in order to present results from the project.
We anticipate that our individual expenses for such travel will be
approximately \$1900 per trip, assuming an average trip length of four
days---an  average of \$900 for airfare and ground transportation, \$600 for
three nights in a hotel, and \$400 for
meals and incidentals (reflecting a standard international per diem of \$100). 
Outside of Canada, there are many different conferences, regular workshops,
and research centres where we expect to send our research to be considered for
presentation.  These venues include the Inland Northwest Philosophy
Conference, the Society for Exact Philosophy, and the annual meetings of the
American Philosophical Association and the Aristotelian Society--Mind Society
Joint Session.

For the second year of the project, we have budgeted a total of \$13,300 to
cover two such trips for each team member, and an additional trip for Hellie,
reflecting the triennial clock of the World
Congress and School on Universal Logic (UNILOG): UNILOG is an ideal venue to
present some of the results of this research.
In the third year of the project, we have budgeted \$17,550 covering two such trips
for each team member, plus an additional trip for Murray to present work at
the annual Italian Conference for Analytic Metaphysics and Ontology, and
an additional trip for
Wilson to present work at the Annual Conference of the Society for the
Metaphysics of Science, to be held in either the
U.K. or in Europe. Finally, the budget in year three reflects a planned stay
for Murray as a visiting researcher at the \emph{Logos} centre for analytic philosophy at the
University of Barcelona, during which stay Murray will workshop in-progress
research related to the project with a number of \emph{Logos} members, both
visiting and permanent.  We anticipate that Murray's expenses for this visit
will be approximately \$2350, reflecting \$900 for airfare and ground
transportation, \$750 for seven nights lodging, and \$700 for meals and
incidentals.  
 
Our budget also includes funds for our research assistants to accompany us on
some of these trips. We have allotted \$1900 for each Toronto and Manitoba
research assistant to travel to at least two conferences outside of Canada
during the term of the project (a total of \$7600 in the second and fourth
years of the project).  In addition we have allotted \$1275 for each research
assistant to attend Canadian conferences in each of year of the project (a
total of \$5100 per year). These conferences will present critical networking
opportunities for our students, and we expect that by the final two years they
will be presenting work of their own related to the research project.

 


\subsubsection*{Other expenses}

We each currently possess well-functioning laptops that are absolutely essential to 
our work. However, we anticipate that each member of the research team
will be in need of a replacement laptop at some point during the term of the
project. We have therefore budgeted \$2500 in each of the first three years
of the project for the purchase of new laptops and perhaps other related
hardware items.

We have also each budgeted \$250 per year to cover the purchase of books and monographs
that will be needed with regular frequency (for a total of \$750 each year).
 While many texts can be obtained through the libraries at our respective
institutions, some will be unavailable or will be required throughout the full
term of the project. We will need to purchase these research materials
ourselves. 

Finally, we have requested funds in each year of the project to support a
partial research leave for Murray. The University of Manitoba affords
successful applicants and co-applicants of SSHRC support the opportunity to
move from a regular 2/2 teaching assignment to a 2/1 assignment during the
granting period. The requested amount of \$6750 reflects the standard
University of Manitoba rate for sessional replacement for one course per year.
Partial teaching leave will greatly facilitate Murray's contribution to the
project research. 
 

\subsubsection*{Funds from other sources}

In addition to the funds requested in the budget, we anticipate applying for
and garnering support from outside sources for the planned workshop activities
discussed in the Plan for Knowledge Mobilization.  For the Toronto workshops
planned in years one and three of the project, Hellie and Wilson will seek
funding in the form of SSHRC Connection Grants. For the Manitoba workshop
planned in year two of the project, Murray will seek funding in the form of a
SSHRC Connection Grant, and a supporting Faculty of Arts Conference Grant from
the University of Manitoba.  


\end{document}

